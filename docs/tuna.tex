% !TeX encoding = UTF-8
% !TeX program = xelatex
% !TeX spellcheck = en_US

\documentclass[degree=master]{thuthesis}
  % 学位 degree:
  %   doctor | master | bachelor | postdoc
  % 学位类型 degree-type:
  %   academic(默认)| professional
  % 语言 language
  %   chinese(默认)| english
  % 字体库 fontset
  %   windows | mac | fandol | ubuntu
  % 建议终版使用 Windows 平台的字体编译

% 论文基本配置,加载宏包等全局配置
\input{thusetup}

\begin{document}

% 封面
\maketitle

% 学位论文指导小组、公开评阅人和答辩委员会名单
% 本科生不需要
\input{data/committee}

% 使用授权的说明
\copyrightpage
% 将签字扫描后授权文件 scan-copyright.pdf 替换原始页面
% \copyrightpage[file=scan-copyright.pdf]

\frontmatter
% abstract.tex
\documentclass[UTF8]{article}

\usepackage{emoji}

\title{abstract}
\author{oeyoews}
\date{2022/08/11}

\begin{document}

\maketitle

\section{\emoji{leaves} Abstract}%
\label{sec:Abstract}

\begin{abstract}
	\LaTeX{} documentation written as \LaTeX! How novel and totally not
	my idea!
\end{abstract}

\end{document}


% 目录
\tableofcontents

% 插图和附表清单
% 本科生的插图索引和表格索引需要移至正文之后、参考文献前
% \listoffiguresandtables  % 插图和附表清单(仅限研究生)
\listoffigures           % 插图清单
\listoftables            % 附表清单

% 符号对照表
\input{data/denotation}

% 正文部分
\mainmatter
\input{data/chap01}
\input{data/chap02}
\input{data/chap03}
\input{data/chap04}

% 其他部分
\backmatter

% 参考文献
\bibliography{ref/refs}  % 参考文献使用 BibTeX 编译
% \printbibliography       % 参考文献使用 BibLaTeX 编译

% 附录
% 本科生需要将附录放到声明之后,个人简历之前
\appendix
% \input{data/appendix-survey}       % 本科生:外文资料的调研阅读报告
% \input{data/appendix-translation}  % 本科生:外文资料的书面翻译
\input{data/appendix}

% 致谢
\input{data/acknowledgements}

% 声明
\statement
% 将签字扫描后的声明文件 scan-statement.pdf 替换原始页面
% \statement[file=scan-statement.pdf]
% 本科生编译生成的声明页默认不加页脚,插入扫描版时再补上;
% 研究生编译生成时有页眉页脚,插入扫描版时不再重复。
% 也可以手动控制是否加页眉页脚
% \statement[page-style=empty]
% \statement[file=scan-statement.pdf, page-style=plain]

% 个人简历、在学期间完成的相关学术成果
% 本科生可以附个人简历,也可以不附个人简历
\input{data/resume}

% 指导教师/指导小组学术评语
% 本科生不需要
\input{data/comments}

% 答辩委员会决议书
% 本科生不需要
\input{data/resolution}

% 本科生的综合论文训练记录表(扫描版)
% \record{file=scan-record.pdf}

\end{document}
